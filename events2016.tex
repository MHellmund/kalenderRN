
% Jährlich wiederkehrende Termine sollten das Makro \year enthalten.
% Yearly events should contain the macro \year.

\event*{\year-01-01}{Neujahr}
\event{\year-03-14}{Albert Einstein (1879)}
\event*{2016-03-25}{Karfreitag}
\event*{2016-03-27}{Ostersonntag}
\event*{2016-03-28}{Ostermontag}
\event*{\year-05-01}{Tag der Arbeit}
\event*{2016-05-05}{Himmelfahrt}
\event{2016-05-15}{Pfingstsonntag}
\event*{2016-05-16}{Pfingstmontag}
\event*{\year-10-03}{Tag der dt. Einheit}
\event{\year-10-09}{John Lennon (1940)}
\event{\year-12-24}{Heiligabend}
\event*{\year-12-25}{1. Weihnachtstag}
\event*{\year-12-26}{2. Weihnachtstag}  

\event*{2016-07-10}{Fußball-EM: Endspiel}[color=DarkTurquoise]

\period{2016-09-12}{2016-10-04}[color=red!30,name=Urlaub]

\period{2016-07-21}{2016-09-02}[color=LightGreen]% Sommerferien in Berlin
\period{2016-02-01}{2016-02-06}[color=LightGreen]% Winterferien in Berlin
\period{2016-03-21}{2016-04-02}[color=LightGreen]% Osterferien in Berlin
\period{2016-05-17}{2016-05-18}[color=LightGreen]% Pfingstferien in Berlin
\period{2016-07-21}{2016-09-02}[color=LightGreen]% Sommerferien in Berlin
\period{2016-10-17}{2016-10-28}[color=LightGreen]% Herbstferien in Berlin
\period{2016-12-23}{2017-01-03}[color=LightGreen]% Weihnachtsferien in Berlin

\endinput
